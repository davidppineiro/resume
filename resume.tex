%%%%%%%%%%%%%%%%%%%%%%%%%%%%%%%%%%%%%%%%%
% Medium Length Professional CV
% LaTeX Template
%
% This template has been downloaded from:
% http://www.LaTeXTemplates.com
%
% Original author:
% Trey Hunner (http://www.treyhunner.com/)
%
% Important note:
% This template requires the resume.cls file to be in the same directory as the
% .tex file. The resume.cls file provides the resume style used for structuring the
% document.
%
%%%%%%%%%%%%%%%%%%%%%%%%%%%%%%%%%%%%%%%%%

%----------------------------------------------------------------------------------------
%	PACKAGES AND OTHER DOCUMENT CONFIGURATIONS
%----------------------------------------------------------------------------------------

\documentclass{resume} % Use the custom resume.cls style

\usepackage[left=0.75in,top=0.6in,right=0.75in,bottom=0.6in]{geometry} % Document margins

\name{Maria A. Kazandjieva} % Your name
%\address{} % Your address
%\address{(000)~$\cdot$~111~$\cdot$~1111 \\ john@smith.com} % Your phone number and email
\address{mariakaz@cs.stanford.edu \\ http://sing.stanford.edu/maria} 

\begin{document}

\vspace{2em}

%----------------------------------------------------------------------------------------
%	EDUCATION SECTION
%----------------------------------------------------------------------------------------

\begin{rSection}{Areas of Interest}
energy-efficiency in computing systems, wireless sensor networks, power data analysis
\end{rSection}

\vspace{1em}

\begin{rSection}{Education}

{\bf Stanford University} \hfill {\em September 2007 --- present (Sept 2013)} \\ 
Ph.D. Candidate in Computer Science  \hfill {\em Defense:
April 29, 2013} \\ 
M.S. in Computer Science \hfill {\em June 2009}  \\

{\bf Mount Holyoke College} \hfill {\em September 2003 --- December 2006} \\ 
B.A. in Computer Science, Magna Cum Laude \\
Minor in Mathematics  \\


\end{rSection}

%----------------------------------------------------------------------------------------
%	WORK EXPERIENCE SECTION
%----------------------------------------------------------------------------------------

\begin{rSection}{Experience}

\begin{rSubsection}{Stanford University}{September 2007 - Present}{Ph.D. Candidate}{Stanford, CA}

\item {\bf Anyware} -- a low-power hybrid compute architecture to combine the benefits of thin clients with the local compute resources of PCs.

\item {\bf PowerNet} --  a building-scale sensing infrastructure for measuring and understanding the energy usage of compute systems. Over 200 plug-level meters track per-device power draw consumption, while software monitors gather utilization data. 

\item {\bf Contact Graphs} -- design and setup a 1000-node low-power wireless network to track the real-world social contacts within a high school.The collected data is used to improve simulation of influenza spread during epidemi.

\item {\bf Wireless Links} --  understand effects of low-level wireless network properties such as temporal correlation of packet reception. Investigate affect of such bursitness on TCP.


\end{rSubsection}

%------------------------------------------------

\begin{rSubsection}{Princeton University}{Feb - Aug 2007}{Research Staff}{Princeton, NJ}


\item {\bf Mars} -- Evaluate efficient interprocess communication methods for mobile phones

\item {\bf SARANA} -- a Spatially Aware, Resource Aware Network Architecture that examines  to leverage  existing mobile computing infrastructures to run large scale, spatially aware applications.



\end{rSubsection}

%------------------------------------------------

\begin{rSubsection}{Princeton University}{June - Aug 2006 }{Research Intern}{Princeton, NJ}

\item Study resource price formation and effect on resource sharing in mobile networks, including lightweight economic models

\end{rSubsection}

\end{rSection}

%----------------------------------------------------------------------------------------
%	PUBLICATIONS
%----------------------------------------------------------------------------------------
\newpage


\begin{rSection}{Publications}
\begin{enumerate}

\item 
Omprakash Gnawali, Rodrigo Fonseca, Kyle Jamieson, {\it Maria Kazandjieva,} David Moss,  Phil Levis

\vspace{-0.7em}
{\small
{\bf CTP: An Efficient, Robust,  Reliable Collection Tree Protocol for Wireless Sensor Networks}
}

\vspace{-0.7em}
To appear in the ACM Transactions on Sensor Networks (TOSN), 2013.

\item {\it Maria Kazandjieva,} Brandon Heller, Omprakash Gnawali, Philip Levis, and Christos Kozyrakis. 

\vspace{-0.7em}
{\bf Measuring and Analyzing the Energy Use of Enterprise Computing Systems.} 

\vspace{-0.7em}
To appear in the Journal of Sustainable Computing, 2013. 

\item {\it Maria Kazandjieva,} Brandon Heller, Omprakash Gnawali, Philip Levis, and Christos Kozyrakis. 

\vspace{-0.7em}
{\bf Green Enterprise Computing Data: Assumptions and Realities.}

\vspace{-0.7em}
In Proceedings of the Third International Green Computing Conference (IGCC), 2012. 

\item Marcel Salath\'{e}, {\it Maria Kazandjieva,} Jung Woo Lee, Phil Levis, Marcus Feldman, and James Jones. 

\vspace{-0.7em}
{\bf A High-Resolution Human Contact Network for Infectious Disease Transmission.}

\vspace{-0.7em}
In Proceedings of the National Academy of Sciences (PNAS), December 13, 2010. 

\item Jung Il Choi, Mayank Jain, {\it Maria  Kazandjieva,} and Philip Levis. 

\vspace{-0.7em}
{\bf Granting Silence to Avoid Wireless Collisions.} 

\vspace{-0.7em}
In Proceedings of the 18th IEEE International Conference on Network Protocols (ICNP), 2010

\item {\it Maria Kazandjieva,} Jung Woo Lee, Marcel Salathe, Marcus Feldman, James Jones, Philip Levis.

\vspace{-0.7em}
{\bf Experiences in Measuring a Human Contact Network for Epidemiology Research.}  

\vspace{-0.7em}
ACM Workshop on Hot Topics in Embedded Networked Sensors (HotEmNets), 2010.

\item Jung Il Choi, {\it Maria Kazandjieva,} Mayank Jain, and Philip Levis. 

\vspace{-0.7em}
{\bf The Case for a Network Protocol Isolation Layer.} 

\vspace{-0.7em}
In Proceedings of the 7th  Conference on Embedded Networked Sensor Systems (SenSys), 2009.

\item {\it Maria Kazandjieva,} Brandon Heller, Philip Levis, and Christos Kozyrakis.

\vspace{-0.7em}
{\bf Energy Dumpster Diving.} 

\vspace{-0.7em}
In the Second Workshop on Power Aware Computing (HotPower), 2009.

\item Yang Chen, Omprakash Gnawali, {\it Maria Kazandjieva,} Philip Levis, and John Regehr.

\vspace{-0.7em}
{\bf Surviving Sensor Network Software Faults.} 

\vspace{-0.7em}
In Proceedings of the 22nd ACM Symposium on Operating System Principles (SOSP), 2009.

\item Kannan Srinivasan, {\it Maria Kazandjieva,} Saatvik Agarwal, and Philip Levis. 

\vspace{-0.7em}
{\bf The Beta Factor: Measuring Wireless Link Burstiness.} 

\vspace{-0.7em}
Proceedings of the 6th ACM Conference on Embedded Networked Sensor Systems (SenSys), 2008.

\item {\it Maria Kazandjieva} and Margaret Martonosi.

\vspace{-0.7em}
{\bf Mars: Portable and Efficient Interprocess Communication for Cellular Phones.} 

\vspace{-0.7em}
Sensing on Everyday Mobile Phones in Support of Participatory Research, SenSys, 2007.


\end{enumerate}
\end{rSection}

%----------------------------------------------------------------------------------------
%	POSTERS AND DEMOS
%----------------------------------------------------------------------------------------

\begin{rSection}{Posters and Demos}

\begin{enumerate}

\item {\it Maria Kazandjieva,} Omprakash Gnawali, and Philip Levis. 

\vspace{-0.7em}
{\bf Visualizing Sensor Network Data with Powertron.}

\vspace{-0.7em}
In proceedings of the 8th ACM Conference on Embedded Networked Systems (SenSys) 2010

\item Kannan Srinivasan, {\it Maria Kazandjieva,} Mayank Jain, Eddie Kim, and Philip Levis. 

\vspace{-0.7em}
{\bf SWAT: Enabling Wireless Network Measurements. }

\vspace{-0.7em}
Demo in ACM SenSys, 2008. Demo in ACM SIGCOMM 2009.

\item {\it Maria Kazandjieva} and Margaret Martonosi. 
\vspace{-0.7em}

{\bf Lightweight Economic Model for Resource Sharing in Wireless Networks.} 

\vspace{-0.7em}
Poster in ACM SIGCSE 2007 Student Research Competition. 

\vspace{-0.7em}
Second Place Award in ACM Grand Finals Undergraduate Category. 

\item Denitsa Tilkidjieva, Nilanjan Banerjee, {\it Maria Kazandjieva,} Sami Rollins, Mark D. Corner. 

\vspace{-0.7em}
{\bf Llama: An Adaptive Strategy for Performing Background Tasks on Mobile Devices. }

\vspace{-0.7em}
Poster in 7th IEEE Workshop on Mobile Computing Systems and Applications (WMCSA 2006). 

\end{enumerate}
\end{rSection}

%----------------------------------------------------------------------------------------
%	AWARDS
%----------------------------------------------------------------------------------------

\begin{rSection}{Awards}

\begin{rSubsection}{}{}{}{}

\item Google Anita Borg Scholarship Finalist, May 2009
\item Stanford Computer Forum Fellowship, September 2007
\item ACM Grand Finals, Undergraduate Category, Second Place, June 2007
\item SIGCSE �07 Student Research Competition, Second Place, March 2007
\item Starr Scholar, C.V. Foundation, February and September 2006
\item CRA-W Distributed Mentor Project, Summer 2005 and Summer 2006
\item Howard Hughes Medical Institute Research Fellowship, Summer 2006
\item Sarah Williston Scholar, October 2005
\item Mildred L. Sanderson Prize for Excellence in Mathematics, May 2004

\end{rSubsection}

\end{rSection}

%----------------------------------------------------------------------------------------
%	TECHNICAL STRENGTHS SECTION
%----------------------------------------------------------------------------------------

\begin{rSection}{Technical Skills}

\begin{tabular}{ @{} >{\bfseries}l @{\hspace{6ex}} l }
Programming Languages & Python, C, nesC, Java, bash scripting, Javascript \\
Operating Systems & Linux, Mac OS, Tiny OS \\
Visualization Frameworks & Protovis, Flare, Matplotlib \\
Data & MySQL, MapReduce \\
Statistical Software & R \\ 
Tools & Git, Vim, VirtualBox, Amazon EC2
\end{tabular}

\end{rSection}



%----------------------------------------------------------------------------------------

\end{document}
